%% Beginning of file 'sample631.tex'
%%
%% Modified 2022 May  
%%
%% This is a sample manuscript marked up using the
%% AASTeX v6.31 LaTeX 2e macros.
%%
%% AASTeX is now based on Alexey Vikhlinin's emulateapj.cls 
%% (Copyright 2000-2015).  See the classfile for details.

%% AASTeX requires revtex4-1.cls and other external packages such as
%% latexsym, graphicx, amssymb, longtable, and epsf.  Note that as of 
%% Oct 2020, APS now uses revtex4.2e for its journals but remember that 
%% AASTeX v6+ still uses v4.1. All of these external packages should 
%% already be present in the modern TeX distributions but not always.
%% For example, revtex4.1 seems to be missing in the linux version of
%% TexLive 2020. One should be able to get all packages from www.ctan.org.
%% In particular, revtex v4.1 can be found at 
%% https://www.ctan.org/pkg/revtex4-1.

%% The first piece of markup in an AASTeX v6.x document is the \documentclass
%% command. LaTeX will ignore any data that comes before this command. The 
%% documentclass can take an optional argument to modify the output style.
%% The command below calls the preprint style which will produce a tightly 
%% typeset, one-column, single-spaced document.  It is the default and thus
%% does not need to be explicitly stated.
%%
%% using aastex version 6.3
\documentclass[preprint2,linenumbers]{aastex631}

\usepackage{amsmath}

%% The default is a single spaced, 10 point font, single spaced article.
%% There are 5 other style options available via an optional argument. They
%% can be invoked like this:
%%
%% \documentclass[arguments]{aastex631}
%% 
%% where the layout options are:
%%
%%  twocolumn   : two text columns, 10 point font, single spaced article.
%%                This is the most compact and represent the final published
%%                derived PDF copy of the accepted manuscript from the publisher
%%  manuscript  : one text column, 12 point font, double spaced article.
%%  preprint    : one text column, 12 point font, single spaced article.  
%%  preprint2   : two text columns, 12 point font, single spaced article.
%%  modern      : a stylish, single text column, 12 point font, article with
%%      wider left and right margins. This uses the Daniel
%%      Foreman-Mackey and David Hogg design.
%%  RNAAS       : Supresses an abstract. Originally for RNAAS manuscripts 
%%                but now that abstracts are required this is obsolete for
%%                AAS Journals. Authors might need it for other reasons. DO NOT
%%                use \begin{abstract} and \end{abstract} with this style.
%%
%% Note that you can submit to the AAS Journals in any of these 6 styles.
%%
%% There are other optional arguments one can invoke to allow other stylistic
%% actions. The available options are:
%%
%%   astrosymb    : Loads Astrosymb font and define \astrocommands. 
%%   tighten      : Makes baselineskip slightly smaller, only works with 
%%                  the twocolumn substyle.
%%   times        : uses times font instead of the default
%%   linenumbers  : turn on lineno package.
%%   trackchanges : required to see the revision mark up and print its output
%%   longauthor   : Do not use the more compressed footnote style (default) for 
%%                  the author/collaboration/affiliations. Instead print all
%%                  affiliation information after each name. Creates a much 
%%                  longer author list but may be desirable for short 
%%                  author papers.
%% twocolappendix : make 2 column appendix.
%%   anonymous    : Do not show the authors, affiliations and acknowledgments 
%%                  for dual anonymous review.
%%
%% these can be used in any combination, e.g.
%%
%% \documentclass[twocolumn,linenumbers,trackchanges]{aastex631}
%%
%% AASTeX v6.* now includes \hyperref support. While we have built in specific
%% defaults into the classfile you can manually override them with the
%% \hypersetup command. For example,
%%
%% \hypersetup{linkcolor=red,citecolor=green,filecolor=cyan,urlcolor=magenta}
%%
%% will change the color of the internal links to red, the links to the
%% bibliography to green, the file links to cyan, and the external links to
%% magenta. Additional information on \hyperref options can be found here:
%% https://www.tug.org/applications/hyperref/manual.html#x1-40003
%%
%% Note that in v6.3 "bookmarks" has been changed to "true" in hyperref
%% to improve the accessibility of the compiled pdf file.
%%
%% If you want to create your own macros, you can do so
%% using \newcommand. Your macros should appear before
%% the \begin{document} command.
%%
\newcommand{\vdag}{(v)^\dagger}
\newcommand\aastex{AAS\TeX}
\newcommand\latex{La\TeX}

\newcommand{\todo}[1]{{\color{red}#1}}
\newcommand{\src}{J1755$-$2527}
\newcommand{\srcfull}{ASKAP J175534.9$-$252749.1}

\DeclareMathOperator{\erfcx}{erfcx}
\DeclareMathOperator{\erf}{erf}
\DeclareMathOperator{\emg}{emg}

\newcommand{\deriv}[2]{\frac{{\rm d}{#1}}{{\rm d}{#2}}}
\newcommand{\dd}[2]{\frac{{\rm d^2}{#1}}{{\rm d}{#2}^2}}
\newcommand{\ToA}[1]{{\rm ToA}_{\rm {#1}}}

%MNRAS style
\newcommand{\figs}{Figs.}
\newcommand{\Fig}{Fig.}
\newcommand{\EFig}{Extended Data Fig.}
\newcommand{\Figs}{Figs.}
\newcommand{\sect}{Section}
\newcommand{\sects}{Sections}
\newcommand{\Sect}{Section}
\newcommand{\Sects}{Sections}
\newcommand{\tab}{Table}
\newcommand{\tabs}{Tables}
\newcommand{\Tab}{Table}
\newcommand{\Tabs}{Tables}
\newcommand{\eqn}{equation}
\newcommand{\eqns}{equations}
\newcommand{\Eqn}{Equation}
\newcommand{\Eqns}{Equations}
\newcommand{\etal}{et~al.}

\defcitealias{2024MNRAS.535..909D}{Paper~I}

%% Reintroduced the \received and \accepted commands from AASTeX v5.2
%\received{March 1, 2021}
%\revised{April 1, 2021}
%\accepted{\today}

%% Command to document which AAS Journal the manuscript was submitted to.
%% Adds "Submitted to " the argument.
%\submitjournal{PSJ}

%% For manuscript that include authors in collaborations, AASTeX v6.31
%% builds on the \collaboration command to allow greater freedom to 
%% keep the traditional author+affiliation information but only show
%% subsets. The \collaboration command now must appear AFTER the group
%% of authors in the collaboration and it takes TWO arguments. The last
%% is still the collaboration identifier. The text given in this
%% argument is what will be shown in the manuscript. The first argument
%% is the number of author above the \collaboration command to show with
%% the collaboration text. If there are authors that are not part of any
%% collaboration the \nocollaboration command is used. This command takes
%% one argument which is also the number of authors above to show. A
%% dashed line is shown to indicate no collaboration. This example manuscript
%% shows how these commands work to display specific set of authors 
%% on the front page.
%%
%% For manuscript without any need to use \collaboration the 
%% \AuthorCollaborationLimit command from v6.2 can still be used to 
%% show a subset of authors.
%
%\AuthorCollaborationLimit=2
%
%% will only show Schwarz & Muench on the front page of the manuscript
%% (assuming the \collaboration and \nocollaboration commands are
%% commented out).
%%
%% Note that all of the author will be shown in the published article.
%% This feature is meant to be used prior to acceptance to make the
%% front end of a long author article more manageable. Please do not use
%% this functionality for manuscripts with less than 20 authors. Conversely,
%% please do use this when the number of authors exceeds 40.
%%
%% Use \allauthors at the manuscript end to show the full author list.
%% This command should only be used with \AuthorCollaborationLimit is used.

%% The following command can be used to set the latex table counters.  It
%% is needed in this document because it uses a mix of latex tabular and
%% AASTeX deluxetables.  In general it should not be needed.
%\setcounter{table}{1}

%%%%%%%%%%%%%%%%%%%%%%%%%%%%%%%%%%%%%%%%%%%%%%%%%%%%%%%%%%%%%%%%%%%%%%%%%%%%%%%%
%%
%% The following section outlines numerous optional output that
%% can be displayed in the front matter or as running meta-data.
%%
%% If you wish, you may supply running head information, although
%% this information may be modified by the editorial offices.
%\shorttitle{AASTeX v6.3.1 Sample article}
%\shortauthors{Schwarz et al.}
%%
%% You can add a light gray and diagonal water-mark to the first page 
%% with this command:
%% \watermark{text}
%% where "text", e.g. DRAFT, is the text to appear.  If the text is 
%% long you can control the water-mark size with:
%% \setwatermarkfontsize{dimension}
%% where dimension is any recognized LaTeX dimension, e.g. pt, in, etc.
%%
%%%%%%%%%%%%%%%%%%%%%%%%%%%%%%%%%%%%%%%%%%%%%%%%%%%%%%%%%%%%%%%%%%%%%%%%%%%%%%%%
%\graphicspath{{./}{figures/}}
%% This is the end of the preamble.  Indicate the beginning of the
%% manuscript itself with \begin{document}.

\begin{document}

\title{Confirmation of \srcfull{} as a long period radio transient}

%% LaTeX will automatically break titles if they run longer than
%% one line. However, you may use \\ to force a line break if
%% you desire. In v6.31 you can include a footnote in the title.

%% A significant change from earlier AASTEX versions is in the structure for 
%% calling author and affiliations. The change was necessary to implement 
%% auto-indexing of affiliations which prior was a manual process that could 
%% easily be tedious in large author manuscripts.
%%
%% The \author command is the same as before except it now takes an optional
%% argument which is the 16 digit ORCID. The syntax is:
%% \author[xxxx-xxxx-xxxx-xxxx]{Author Name}
%%
%% This will hyperlink the author name to the author's ORCID page. Note that
%% during compilation, LaTeX will do some limited checking of the format of
%% the ID to make sure it is valid. If the "orcid-ID.png" image file is 
%% present or in the LaTeX pathway, the OrcID icon will appear next to
%% the authors name.
%%
%% Use \affiliation for affiliation information. The old \affil is now aliased
%% to \affiliation. AASTeX v6.31 will automatically index these in the header.
%% When a duplicate is found its index will be the same as its previous entry.
%%
%% Note that \altaffilmark and \altaffiltext have been removed and thus 
%% can not be used to document secondary affiliations. If they are used latex
%% will issue a specific error message and quit. Please use multiple 
%% \affiliation calls for to document more than one affiliation.
%%
%% The new \altaffiliation can be used to indicate some secondary information
%% such as fellowships. This command produces a non-numeric footnote that is
%% set away from the numeric \affiliation footnotes.  NOTE that if an
%% \altaffiliation command is used it must come BEFORE the \affiliation call,
%% right after the \author command, in order to place the footnotes in
%% the proper location.
%%
%% Use \email to set provide email addresses. Each \email will appear on its
%% own line so you can put multiple email address in one \email call. A new
%% \correspondingauthor command is available in V6.31 to identify the
%% corresponding author of the manuscript. It is the author's responsibility
%% to make sure this name is also in the author list.
%%
%% While authors can be grouped inside the same \author and \affiliation
%% commands it is better to have a single author for each. This allows for
%% one to exploit all the new benefits and should make book-keeping easier.
%%
%% If done correctly the peer review system will be able to
%% automatically put the author and affiliation information from the manuscript
%% and save the corresponding author the trouble of entering it by hand.

%\correspondingauthor{August Muench}
%\email{greg.schwarz@aas.org, gus.muench@aas.org}

\author[0000-0001-6114-7469]{Samuel J. McSweeney}
\affiliation{International Centre for Radio Astronomy Research, Curtin University, Bentley, WA 6102, Australia}

\author[0000-0002-5119-4808]{Natasha Hurley-Walker}
\affiliation{International Centre for Radio Astronomy Research, Curtin University, Bentley, WA 6102, Australia}

\author{Dougal Dobie}
\affiliation{Sydney Institute for Astronomy, School of Physics, The University of Sydney, NSW 2006, Australia}
\affiliation{ARC Centre of Excellence for Gravitational Wave Discovery (OzGrav), Hawthorn, VIC 3122, Australia}

\author[0009-0003-0996-9176]{Csanad Horv\'{a}th}
\affiliation{International Centre for Radio Astronomy Research, Curtin University, Bentley, WA 6102, Australia}

\author{Angie Waszewski}
\affiliation{International Centre for Radio Astronomy Research, Curtin University, Bentley, WA 6102, Australia}

\author{John Morgan}
\affiliation{International Centre for Radio Astronomy Research, Curtin University, Bentley, WA 6102, Australia}

\author{Ziteng Wang}
\affiliation{International Centre for Radio Astronomy Research, Curtin University, Bentley, WA 6102, Australia}

%\collaboration{20}{(AAS Journals Data Editors)}

% Example with "altaffiliation"
%\author{Amy Hendrickson}
%\altaffiliation{AASTeX v6+ programmer}
%\affiliation{TeXnology Inc.}

%% Note that the \and command from previous versions of AASTeX is now
%% depreciated in this version as it is no longer necessary. AASTeX 
%% automatically takes care of all commas and "and"s between authors names.

%% AASTeX 6.31 has the new \collaboration and \nocollaboration commands to
%% provide the collaboration status of a group of authors. These commands 
%% can be used either before or after the list of corresponding authors. The
%% argument for \collaboration is the collaboration identifier. Authors are
%% encouraged to surround collaboration identifiers with ()s. The 
%% \nocollaboration command takes no argument and exists to indicate that
%% the nearby authors are not part of surrounding collaborations.

%% Mark off the abstract in the ``abstract'' environment. 
\begin{abstract}

  [Abstract goes here\dots]

\end{abstract}

%% Keywords should appear after the \end{abstract} command. 
%% The AAS Journals now uses Unified Astronomy Thesaurus concepts:
%% https://astrothesaurus.org
%% You will be asked to selected these concepts during the submission process
%% but this old "keyword" functionality is maintained in case authors want
%% to include these concepts in their preprints.
%\keywords{Classical Novae (251) --- Ultraviolet astronomy(1736) --- History of astronomy(1868) --- Interdisciplinary astronomy(804)}

%% From the front matter, we move on to the body of the paper.
%% Sections are demarcated by \section and \subsection, respectively.
%% Observe the use of the LaTeX \label
%% command after the \subsection to give a symbolic KEY to the
%% subsection for cross-referencing in a \ref command.
%% You can use LaTeX's \ref and \label commands to keep track of
%% cross-references to sections, equations, tables, and figures.
%% That way, if you change the order of any elements, LaTeX will
%% automatically renumber them.
%%
%% We recommend that authors also use the natbib \citep
%% and \citet commands to identify citations.  The citations are
%% tied to the reference list via symbolic KEYs. The KEY corresponds
%% to the KEY in the \bibitem in the reference list below. 

\section{Introduction} \label{sec:introduction}

\srcfull{} (hereafter, \src{}) is a transient radio source discovered in the Variables and Slow Transients (VAST) survey \citep[][hereafter \citetalias{2024MNRAS.535..909D}]{2024MNRAS.535..909D}.
A single, highly polarized pulse was observed, lasting approximately 2 minutes, exhibiting a steep spectrum ($\alpha = -3.1$).
Extensive follow up in new and archival data at multiple wavelengths turned up no other pulses.
Despite this, the authors concluded that \src{} was most likely a long period transient (LPT), a newly emerging class of radio source with periods ranging from minutes to hours.

Although optical follow-up has confirmed that at least some LPTs are white dwarfs with main sequence companions \citep{2024arXiv240811536D,2025arXiv250103315R}, it remains unclear if this model can account for all known LPTs \citep[e.g.][]{2022ApJ...940...72R,Lee2025}.
Unfortunately, optical follow-up is not possible for LPTs and LPT candidates which are obscured by Galactic foregrounds.
This is almost certainly the case for \src{}, which is situated very close to the Galactic plane ($b = -0\overset{\circ}{.}12$), and has a dispersion measure (DM) derived distance of ${\sim}4.7\,$kpc \citepalias{2024MNRAS.535..909D}; unsurprisingly, optical follow-up at \src{}'s location did not turn up any source down to ${\sim}23$ AB mag.
Without more detections in the radio or at other wavelengths, it seemed unlikely that the nature of \src{} could ever be resolved.

In this short paper, we present new radio observations (spanning frequencies from $154\,$MHz to $1.7\,$GHz) in which \src{} was re-detected, confirming that it is indeed an LPT with a period of ${\sim}1.16\,$hours.
In \S\ref{sec:observations}, we describe the observations themselves, some of which were taken as part of independent surveys, and some of which were taken as dedicated follow-up of \src{} once it became apparent that it was still active.
In \S\ref{sec:analysis} we present a timing analysis of the new detections, as well as revised estimates of the DM and the scattering timescale.
We also include new polarization detections in which the polarisation angle (PA) behavior differs significantly from the originally detected pulse.
In \S\ref{sec:discussion} we revisit the question of the nature of \src{}, arguing in favor of a white dwarf interpretation.
The main conclusions are summarized in \S\ref{sec:conclusions}.

\section{Observations} \label{sec:observations}

\subsection{MWA} \label{sec:mwa}

\src{} was detected blindly with the Murchison Widefield Array \citep[MWA;][]{Tingay2013} in observations taken between June--September 2024 under the ``Galactic Plane Monitor'', project code G0080. The prime objective of the campaign was to search for long-period transients by scanning the Galactic plane over $285^\circ < l < 65^\circ$, $|b|<\pm15^\circ$ on a bi-weekly cadence at 185--215\,MHz (see Methods of \citet{2023Natur.619..487H}); a full description will be released by Hurley-Walker et al., in prep. We also observed over June--September 2022 but made no detections in these data.

We re-imaged the 2024 data at the location of \src{} using \textsc{WSClean}, masking \src{} itself, forming a deep model of the sky. After subtracting this from the visibilities, we phase-rotated to \src{}, and averaged the baseline data to produce a dynamic spectrum.
% SM to write a 'joining sentence' to explain that the timing analysis was a bit more complicated because of the low frequencies and refer the reader to the relevant section -- SM (memo to self): Not that complicated because DM + tau_sc can be ignored if making ToA predictions solely based on pulses at the target frequency.

% SM: commenting the following out because it's a repeat of above...?
%The GPM also conducted observations of this region in June--September 2022, but \src{} was not detected, either by a blind search, nor by a re-examination of the data following the methods above.

% NWH to add what went into the followup observations: D0042 (both 185 and 200 MHz). SM

% ** ANGIE **
% Plz add a short paragraph here with the essential details of the IPS observation included in this paper: G0060, upper half-band only, how the images were made. I'll (SM) add a few sentences about how I selected that observation in the first place.
An additional MWA observation made as part of the MWA Interplanetary Scintillation (IPS) Survey, project code G0060, has also been included in this present study. 
A full description of how MWA IPS observations are scheduled and processed can be found in Section~2 of \citet{Morgan2022}; here, we provide a brief overview.
The two adjacent 200-second observations included in this work are a sub-section of a full 10-minute near-Sun observation (30$^{\circ}$ elongation).
These observations are taken over two equal bands of frequency, centered on approximately 80\,MHz and 160\,MHz. For this study, only the upper band has been processed. 
IPS observations require a short time integration of 0.5\,s, but can allow for a larger frequency resolution of 40\,kHz, which is then averaged to an even higher 160\,kHz during imaging. % SM: TODO: describe various scrunching I did to produce light curve in pulsestack
The observation was self-calibrated against a sky model based on the GLEAM survey \citep{Hurley-Walker2017} using \textsc{mwa\_hyperdrive}\footnote{\url{https://github.com/MWATelescope/mwa_hyperdrive}}, and then imaged using \textsc{WSClean}.
% NHW: then phase-rotated and averaged to produce a dynamic spectrum?




\subsection{ASKAP} \label{sec:askap}

% NHW to update this in light of new historical discovery of a dim pulse in SB49153.

Paper~1 presented an exhaustive search of the ASKAP data available at the time of writing, comprising 60\,h of data, and yielding only a single detection of \src{} (in Scheduling Block ID 47253). Subsequent to the writing of Paper~1, a further epoch of the VAST survey was performed (SBID 63600) in which \src{} produced a detectable pulsation.

With the re-detection of the source in the MWA data, and the calculation of an ephemeris (\Sect~\ref{sec:analysis}), we re-reduced the data in which we would predict that \src{} would appear. We made one further detection, in SBID~49153, at a relatively low significance ($\sim5\sigma$) compared to the blind detections. We followed a similar procedure as the above to generate dynamic spectra for each of the observations, using a minimum baseline length of 500\,m to suppress uncleaned Galactic diffuse emission.


\subsection{MeerKAT} \label{sec:meerkat}

MeerKAT observations were undertaken on 2024-10-19, under proposal code 
DDT-20241015-NH-01 and capture block 1729341386, using the UHF band spanning 544 -- 1088\,MHz. As well as correlator observations undertaken at 2-s/132.812-kHz resolution, we also employed the Pulsar Timing User Supplied Equipment \citep[PTUSE;][]{2020PASA...37...28B} in search mode, using 37.45-$\mu$s sampling. The observation included typical bandpass, polarisation, and phase calibrators, as well as phase-up and test pulsar observations. \src{} was tracked from 13:54:36 to 14:04:35 UTC.

The correlator data were calibrated using the standard SARAO SDP calibration pipeline, and imaged using \textsc{WSClean}. We formed a deep model of the sources within the primary beam, excepting \src{}, and subtracted this from the visibilities. We then averaged the visibilities, excluding all baselines shorter than 150\,m, to produce a dynamic spectrum of the source, shown in \Fig~\ref{fig:scattering_DM}.

We also re-imaged the original 2023-05-28 L-band MeerKAT observation described in Paper~1, which was taken under project code DDT-20230525-DD-01 under capture block 1685306788. We performed a similar processing of the data, and identified four pulses with TOAs consistent with an ephemeris derived from the other data. The dynamic spectrum of one of the brightest pulses is shown in \Fig~\ref{fig:scattering_DM}.

A summary of all the observations used in this paper is given in \Tab~\ref{tbl:obs}.

\begin{deluxetable*}{lllccc}
  \tablecaption{Observations summary\label{tbl:obs}}
  \tablehead{
    \colhead{Date range} & \colhead{Telescope} & Project & \colhead{$\nu$} & \colhead{$\Delta\nu$} & \colhead{Number of} \\ 
    \colhead{(MJD)} & & & \colhead{(MHz)} & \colhead{(MHz)} & \colhead{pulses}
  } 
  \startdata
  59965 & ASKAP\tablenotemark{a} & SB47253 & 887.5 & 288 & 1 \\
  59966 & MWA & G0060 & 161.92 & 15.36 & 1 \\
  %60084 & MWA & G0057 & 154.24 & 30.72 & $\le$ 1 \\
  60040  & ASKAP & SB49153 & 887.5 & 288 & 1 \\
  60092 - 60093 & MeerKAT & DDT-20230525-DD-01 & 1284 & 856 & 4 \\
  60463 - 60563 & MWA & G0080 & 200.32 & 30.72 & 16 \\
  60503 & ASKAP & SB63600 & 887.5 & 288 & 1 \\
  60572 & MWA & D0042 & 184.96 & 30.72 & 4 \\
  60592 - 60602 & MWA & D0042 & 200.32 & 30.72 & 17 \\
  60602 & MeerKAT & DDT-20241015-NH-01 & 812.8 & 495.3 & 1 \\
  \enddata
  \tablenotetext{a}{Original discovery \citepalias{2024MNRAS.535..909D}}
\end{deluxetable*}

%\begin{deluxetable*}{cccccc}
%\tablecaption{Pulse detections and non-detections}
%\tablehead{
%\colhead{Obs start} & \colhead{Duration} & \colhead{Telescope} & \colhead{$\nu$} & %\colhead{$\Delta\nu$} & \colhead{Detected?} \\ 
%\colhead{time (UTC)} & \colhead{(s)} & & \colhead{(MHz)} & \colhead{(MHz)} &
%} 
%\startdata
%2024-09-19 09:25:02 & 592\tablenotemark{a} & MWA & 184.96 & 30.72 & Y \\ % 1410773120 and 1410773416
%2024-09-19 10:34:06 & 592\tablenotemark{a} & MWA & 184.96 & 30.72 & Y \\ % 1410777264 and 1410777560
%2024-09-19 12:57:10 & 296 & MWA & 184.96 & 30.72 & Y \\ % 1410785848
%2024-09-19 14:06:14 & 296 & MWA & 184.96 & 30.72 & Y \\ % 1410789992
%2024-10-09 10:55:50 & 296 & MWA & 200.32 & 30.72 & Y \\ % 1412506568
%2024-10-09 12:05:34 & 296 & MWA & 200.32 & 30.72 & Y \\ % 1412510752
%2024-10-10 11:21:10 & 296 & MWA & 200.32 & 30.72 & Y \\ % 1412594488
%2024-10-10 12:30:54 & 296 & MWA\tablenotemark{b} & 200.32 & 30.72 & Y \\ % 1412598672
%2024-10-11 10:36:46 & 296 & MWA & 200.32 & 30.72 & Y \\ % 1412678224
%2024-10-11 11:46:30 & 296 & MWA & 200.32 & 30.72 & Y \\ % 1412682408
%2024-10-12 11:02:06 & 296 & MWA & 200.32 & 30.72 & Y \\ % 1412766144
%2024-10-12 12:11:50 & 296 & MWA & 200.32 & 30.72 & Y \\ % 1412770328
%2024-10-13 10:17:42 & 296 & MWA & 200.32 & 30.72 & Y \\ % 1412849880
%2024-10-13 11:27:26 & 296 & MWA & 200.32 & 30.72 & Y \\ % 1412854064
%2024-10-14 10:43:02 & 296 & MWA & 200.32 & 30.72 & Y \\ % 1412937800
%2024-10-14 11:52:46 & 296 & MWA & 200.32 & 30.72 & Y \\ % 1412941984
%2024-10-16 10:23:58 & 296 & MWA & 200.32 & 30.72 & Y \\ % 1413109456
%2024-10-16 11:33:42 & 296 & MWA & 200.32 & 30.72 & Y \\ % 1413113640
%2024-10-17 11:59:02 & 296 & MWA & 200.32 & 30.72 & Y \\ % 1413201560
%2024-10-19 10:30:14 & 296 & MWA & 200.32 & 30.72 & Y \\ % 1413369032
%2024-10-19 11:39:58 & 296 & MWA & 200.32 & 30.72 & Y \\ % 1413373216
%2024-10-19 13:54:36 & 601 & MeerKAT\tablenotemark{c} & 812.8 & 247.6 & Y \\ % 1413381294
%\enddata
%\tablenotetext{a}{Two contiguous 296-second MWA observations were combined}
%\tablenotetext{b}{Too much RFI to use in downstream analysis}
%\tablenotetext{c}{Two components in pulse}
%\end{deluxetable*}


\section{Analysis \& results} \label{sec:analysis}

%[Reference for scattering kernels:] \citep{10.1093/mnras/stw1724}.

\subsection{Pulse morphology}

All observed pulses consist of a single burst, approximately Gaussian in shape, with the sole exception of the MeerKAT pulse observed on 2024-10-19, for which two bursts were observed separated by ${\sim}125\,$s.
Generally, the low-frequency (MWA) pulses tend
\begin{itemize}
    \item to be wider than their higher frequency counterparts,
    \item to exhibit an asymmetry resembling a scatter-broadened tail, and
    \item to have minimal pulse-to-pulse morphological variation.
\end{itemize}

The low-frequency asymmetry is consistent with a scattering time scale of a few tens of seconds at MWA frequencies, which scales to a few tens of milliseconds at 1\,GHz.
This is in broad agreement with the model predictions of NE2001 ($\tau_{\rm sc,1GHz} = 0.066\,$s) and YMW16 ($\tau_{\rm sc,1 GHz} = 0.075\,$s) at the source's location assuming the DM value of 710\,pc\,cm$^{-3}$ reported in \citepalias{2024MNRAS.535..909D}.

To account for the asymmetry, we fit exponentially modified Gaussians to each pulse with the form,
\begin{equation}
  \begin{aligned}
        \emg(t) &= A \frac{\sigma}{\tau}\sqrt{\frac{\pi}{2}}
                   \exp\left(-\frac12 \left( \frac{t - \mu}{\tau} \right)^2 \right) \cdot {} \\
                            &\qquad \erfcx \left(\frac{1}{\sqrt{2}} \left(\frac{\sigma}{\tau} 
                             - \frac{t - \mu}{\tau} \right) \right),
  \end{aligned}
  \label{eqn:emg}
\end{equation}
where $A$, $\mu$, and $\sigma$ are the height, location, and scale of an unscattered Gaussian pulse, and the scattering time scale was fixed at
\begin{equation}
      \tau(\nu) = 0.07\, \left(\frac{\nu}{\rm GHz}\right)^{-4} \, {\rm s},
\end{equation}
according to each observation's centre frequency.
A pulse stack containing all observed pulses and the corresponding unscattered pulses derived from the fits is shown in Fig. \ref{fig:pulsestack}.

\begin{figure*}[tp]
      \centering
          \includegraphics[width=0.95\linewidth]{pulsestack.pdf}
              \caption{Pulsestack of barycentred, dedispersed profiles from the ephemeris. The dashed red vertical line marks zero phase. Baselines have been fitted and subtracted from each lightcurve. The blue dashed lines show (normalised) fitted Gaussians without scattering (see main text for details).}
                  \label{fig:pulsestack}
\end{figure*}

For the MeerKAT pulse with two components, we fit only the brighter component.

\subsection{Timing solution} \label{sec:timing}

Because the large scattering time scale at low frequencies significantly distorts the pulse shapes, common methods for defining ToAs can potentially result in systematic errors of up to tens of seconds, as discussed in detail in Appendix \ref{app:scattering_dm}.
These systematic errors set up a degeneracy between DM and scattering time scales that is not easily disambiguated.

This is demonstrated in Fig. \ref{fig:stacked_spectra}, which shows stacked MWA and MeerKAT spectra.
The cyan lines show the expected ToAs across the whole observed frequency range if one assumes only a DM measured at higher frequencies, where scattering is negligible.
The scattering predicted by Galactic electron density models, however, is sufficient to account for the apparent extra delay, as is evidenced by the relatively good agreement with the modelled unscattered pulses with the expected ToAs derived from the ephemeris, illustrated in the pulsestack shown in Fig. \ref{fig:pulsestack}.

\begin{figure}[th]
      \centering
          \includegraphics[width=0.95\linewidth]{stacked_spectra.pdf}
              \caption{Stacked (semi-transparent) dynamic spectra, where each rectangle indicates the observing campaign whose spectra were barycentred and folded according to the ephemeris. The spectra have not been dedispersed, but the cyan dashed lines indicate where the two components seen in the MeerKAT (2024) observation would appear due to dispersion. The white curve in the top panel is the dedispersed profile of the MeerKAT (2024) observation, and the white curves in the bottom panel are the same profile subjected to scattering with $\tau_{\rm sc} = \tau_{\rm sc,1\,GHz} (\nu/{\rm 1\,GHz})^{-4}$ at frequencies 175, 195, and 210 MHz, where $\tau_{\rm sc,1\,GHz} = 0.07\,$s.}
                  \label{fig:stacked_spectra}
\end{figure}

We therefore identified the times of arrival (ToAs) with the $\mu$ parameter in the fits given in Eq. \eqref{eqn:emg}, which represents the location of the assumed Gaussian pulse before scattering.
Because the scattering time scale could not be reliably measured for all individual pulses, we obtained ToAs by assuming a fixed scattering time of $\tau_{\rm sc,1 GHz} = 0.07\,$s.

%\begin{deluxetable*}{ccccc}
%  \tablecaption{Times of arrival and pulse properties\label{tbl:toas}}
%  \tablehead{
%    \colhead{Telescope} & \colhead{$\nu$} & \colhead{ToA} & \colhead{Fluence} & \colhead{$\sigma$} \\ 
%    & \colhead{(MHz)} & \colhead{(MJD)} & \colhead{($10^3$\,Jy\,s)} & \colhead{(s)}
%  } 
%  \startdata
%ASKAP & 888 & 59965.0429358(45) & 0.48(2) & 29.1(4) \\
%MWA & 170 & 59966.061584(40) & 31.7(7) & 40(7) \\
%ASKAP & 888 & 60040.913447(32) & 0.03(1) & 25(3) \\
%MeerKAT & 1284 & 60092.899548(10) & 0.003(3) & 13(1) \\
%MeerKAT & 1284 & 60092.947966(10) & 0.003(3) & 12(1) \\
%MeerKAT & 1284 & 60092.996466(11) & 0.005(3) & 16(1) \\
%MeerKAT & 1284 & 60093.044919(12) & 0.005(4) & 16(1) \\
%MWA & 200 & 60481.637466(65) & 3.0(2) & 24(8) \\
%MWA & 200 & 60485.658977(33) & 1.18(7) & 16(4) \\
%MWA & 200 & 60489.632266(20) & 1.65(6) & 14(2) \\
%MWA & 200 & 60492.588020(24) & 3.12(9) & 14(3) \\
%MWA & 200 & 60496.609639(18) & 2.87(8) & 17(2) \\
%MWA & 200 & 60503.587085(14) & 3.40(8) & 17(2) \\
%ASKAP & 1032 & 60503.634781(17) & 0.09(1) & 16.6(5) \\
%MWA & 200 & 60506.542809(33) & 1.35(6) & 9(4) \\
%MWA & 200 & 60510.564531(11) & 3.08(7) & 16(1) \\
%MWA & 200 & 60519.528812(29) & 3.1(1) & 14(4) \\
%MWA & 200 & 60522.533028(16) & 3.50(8) & 18(2) \\
%MWA & 200 & 60527.524033(14) & 2.89(7) & 16(2) \\
%MWA & 200 & 60530.528339(16) & 2.90(8) & 17(2) \\
%MWA & 200 & 60537.506042(13) & 3.91(9) & 17(2) \\
%MWA & 200 & 60541.479448(47) & 1.48(9) & 14(5) \\
%MWA & 200 & 60544.483761(15) & 3.91(9) & 20(2) \\
%MWA & 200 & 60551.461625(16) & 3.35(8) & 16(2) \\
%MWA & 185 & 60572.395455(15) & 2.86(7) & 15(2) \\
%MWA & 185 & 60572.443943(21) & 3.01(8) & 16(2) \\
%MWA & 185 & 60572.540824(18) & 2.57(8) & 16(2) \\
%MWA & 185 & 60572.589281(31) & 4.3(2) & 21(4) \\
%MWA & 200 & 60592.456765(27) & 1.63(8) & 19(3) \\
%MWA & 200 & 60592.505159(33) & 1.67(9) & 19(4) \\
%MWA & 200 & 60593.474369(26) & 1.08(6) & 15(3) \\
%MWA & 200 & 60593.522901(46) & 1.3(1) & 19(5) \\
%MWA & 200 & 60594.443484(20) & 1.94(8) & 18(2) \\
%MWA & 200 & 60594.491990(33) & 0.55(5) & 12(4) \\
%MWA & 200 & 60595.461118(29) & 1.53(8) & 19(3) \\
%MWA & 200 & 60595.509617(25) & 0.89(5) & 13(3) \\
%MWA & 200 & 60596.430282(20) & 1.55(7) & 16(2) \\
%MWA & 200 & 60596.478721(23) & 1.65(7) & 17(3) \\
%MWA & 200 & 60597.447846(41) & 0.86(7) & 16(5) \\
%MWA & 200 & 60597.496319(30) & 1.82(9) & 19(3) \\
%MWA & 200 & 60599.434635(17) & 1.28(5) & 13(2) \\
%MWA & 200 & 60599.483121(24) & 1.00(5) & 13(3) \\
%MWA & 200 & 60600.500655(31) & 1.24(7) & 16(4) \\
%MWA & 200 & 60602.438964(32) & 1.9(1) & 21(4) \\
%MWA & 200 & 60602.487394(27) & 1.98(9) & 20(3) \\
%MeerKAT & 813 & 60602.5835731(40) & 0.061(7) & 14.9(4)
  %\enddata
%\end{deluxetable*}

These ToAs were then barycentred and fit for both period and DM, resulting in the values given in Table \ref{tbl:ephemeris}.
The residuals are plotted in \ref{fig:pulse_details}, along with the fitted pulse widths and fluences.
We also tried fitting a timing model that included a spin-down parameter, and found that it was consistent with zero spin-down to within $1\sigma$, and so is not included in the ephemeris.

\begin{table}
  \caption{Timing ephemeris for \src{}}
  \label{tbl:ephemeris}
  \begin{tabular}{lc}
    \tableline
    Parameter & Value \\
    \tableline
    Period (s) & $4186.3283 \pm 0.0003$ \\
    PEPOCH (MJD) & $59965.03790 \pm 0.00003$ \\
    DM (pc/cm$^{-3}$) & $764 \pm 25$ \\
    $\tau_{\rm sc,1 GHz}$ (s) & ${\sim}0.07$\tablenotemark{a} \\
    \tableline
  \end{tabular}
 \tablenotetext{a}{Assumed value, degenerate with DM (see main text).}
\end{table}

\begin{figure*}[th]
  \centering
  \includegraphics[width=0.98\linewidth]{pulse_details.pdf}
  \caption{Timing residuals (top panel) and other pulse properties derived from the fits of Eq. \eqref{eqn:emg} to the individual light curves (lower two panels). \dots}
  \label{fig:pulse_details}
\end{figure*}

The measured DM of $764 \pm 25\,{\rm pc}\,{\rm cm}^{-3}$ is slightly higher than, but still consistent with, the in-band measurement of $710^{+200}_{-180}\,{\rm pc}\,{\rm cm}^{-3}$ given in \citetalias{2024MNRAS.535..909D}.
Because we have not attempted to simultaneously fit for both DM and scattering timescale, there will remain some small systematic error on this reported DM measurement.
Breaking the degeneracy is possible in principle with a much higher S/N average profile than what is possible with the current data set.

For planning observations of \src{} at low ($\lesssim 300 MHz$), care should be taken to incorporate scattering into the predicted ToAs.
The $N$th pulse since PEPOCH is predicted to arrive at the Solar System barycentre at
\begin{equation}
    {\rm MJD} = {\rm PEPOCH} + NP + \Delta t_{\rm DM}(\nu) + \Delta t_{\rm sc}(\nu),
\end{equation}
where PEPOCH and the period, $P$, are given in Table \ref{tbl:ephemeris},
\begin{equation}
    \Delta t_{\rm DM}(\nu) = 4148\,{\rm s} \, \left(\frac{\rm DM}{{\rm pc\,cm}^{-3}}\right) \left(\frac{\nu}{\rm MHz}\right)^{-2}
\end{equation}
is the usual dispersion attributed to the interstellar medium, and
\begin{equation}
    \Delta t_{\rm sc}(\nu) = -\sqrt{2}\sigma\erfcx^{-1}\left(\frac{\tau(\nu)}{\sigma}\sqrt{\frac{2}{\pi}}\right) + \frac{\sigma^2}{\tau(\nu)}
    \label{eqn:scattering_delay}
\end{equation}
is the amount that the \emph{peak} of the observed pulse is delayed by scattering.
When calculating this quantity, we recommend using $\sigma = 15\,$s, a typical value for low-frequency pulses (see Fig. \ref{fig:pulse_details}).
At frequencies above $300\,$MHz, $\tau(\nu) \ll \sigma$, and the delay due to scattering, which asymptotically behaves like $\Delta t_{\rm sc}(\nu) \sim \tau$, can be safely neglected.
At frequencies below $150\,$MHz, the asymptotic behaviour
\begin{equation}
    \Delta t_{\rm sc}(\nu) \sim \sigma \sqrt{\ln \left(\frac{\tau^2}{2\pi \sigma^2}\right)} + \frac{\sigma^2}{\tau}
\end{equation}
is accurate to within a few seconds.
At intermediate frequencies ($150\,{\rm MHz} \lesssim \nu \lesssim 300\,{\rm MHz}$), Eq. \eqref{eqn:scattering_delay} should be used directly.

%Following the positive identification of several pulses in the GPM data set, we performed a grid search of periods that would result in an integer number of pulses between two widely spaced pulses.
%A candidate period of $P = 4186.4 \pm 0.1$ was found for which all detections fell within a narrow window of ${\sim}0.05$ of phase.
% We further refined this by attempting to phase connect to Dougal's original pulse, and got $4186.33 \pm 0.02$ s, but this needn't be part of the story, since even the above was enough to motivate the MWA campaign that followed.
%This motivated the follow-up MWA observations (under project code D0042) in which two consecutive pulses ${\sim}4190$ seconds apart were detected without any discernible emission between them, ruling out periods $P/n$, for $n \ge 2$.
%The lever arm afforded by both the GPM and D0042 detections proved sufficient to phase-connect to the original pulse published by \citetalias{2024MNRAS.535..909D}, yielding a refined period of $P = 4186.33 \pm 0.02$~s.

%We then measured times of arrival (ToAs) of individual pulses.
%The ASKAP and MWA pulses were morphologically similar and well fit by a simple Gaussian.

\subsection{Polarization} \label{sec:polarization}

No significant polarised emission was detected in the MWA pulses, as well as the 2023 MeerKAT pulses.
The apparent lack of polarisation in the MeerKAT data set is likely due to low S/N coupled with the source not being not being centred in the beam.
Much more careful analysis would be needed to obtain reliable limits on the polarised components of those pulses.

The apparent lack of linear polarisation in the MWA pulses, however, is expected due to the high RM of ${\sim}961\,{\rm rad}/{\rm m}^2$ \citepalias[as reported in][]{2024MNRAS.535..909D}, causing the pulses to be depolarised within individual channels.
We also report that no significantly circularly polarised emission was seen at MWA frequencies.

The three sufficiently bright pulses for which we report significant polarisation are (1) the original 2023 ASKAP pulse \citepalias{2024MNRAS.535..909D}, (2) the 2024 ASKAP pulse, and (3) the 2024 MeerKAT pulse.
The high S/N of each time bin allowed us to test whether the rotation measure (RM) was consistent across each pulse.
We therefore performed a joint fit of the Stokes Q and U spectra of each time bin to
\begin{equation}
    Q(\nu) = {\rm Re}(\mathcal{L}(\nu)),
    \quad
    U(\nu) = {\rm Im}(\mathcal{L}(\nu)),
    \label{eqn:RM}
\end{equation}
where
\begin{equation}
    \mathcal{L}(\nu)
        = L_{\rm 1\,GHz}\left(\frac{\nu}{\rm 1\,GHz}\right)^{\alpha_L} e^{2i({\rm RM}\,\lambda^2 + \psi_0)},
\end{equation}
$L_{\rm 1\,GHz}$ is the flux density of the linearly polarised component at 1\,GHz, $\alpha_L$ is its spectral index, RM is the rotation measure, $\lambda = c/\nu$ is the wavelength, and $\psi_0$ is the intrinsic polarisation angle.
The RMs and spectral indices of these fits are shown in the top two rows of panels of Fig. \ref{fig:RM}.

\begin{figure*}[th]
    \centering
    \includegraphics[width=0.98\linewidth]{RM.pdf}
    \caption{Polarisation of the 2023 and 2024 ASKAP pulses (left and middle) and the 2024 MeerKAT pulse (right). The top two rows show the RM and $\alpha_{\rm lin}$ parameters fitted to Eq. \eqref{eqn:RM} for each time bin independently. Only bins for which the uncertainty on $\alpha_{\rm lin}$ is less than $1.4$ are shown, a threshold which was found to cleanly eliminate off-pulse bins. The weighted average RMs for each pulse are indicated by the horizontal red lines. The PAs and lightcurves shown in the bottom two rows of panels were produced using only the average RM for the respective pulse.}
    \label{fig:RM}
\end{figure*}

We found that the RM derived for the MeerKAT pulse had the opposite sign, but similar magnitude, to the ASKAP pulses, which is most likely due to a different sign convention being used within the two telescopes' pre-processing software.
The agreement presented in Fig. \ref{fig:RM} was achieved by first negating Stokes Q in the MeerKAT data.
Any reflection in the Q-U plane (e.g. ${\rm Q} \leftrightarrow -{\rm Q}$, ${\rm U} \leftrightarrow -{\rm U}$, or ${\rm Q} \leftrightarrow {\rm U}$) will achieve a similar correction to the sign of the RM, but different choices will introduce different offsets to the PA.
Because of this, the MeerKAT PA values should not be directly compared with the ASKAP values.
The most likely scenario, however, is that the MeerKAT values include an arbitrary offset of $n \times 45^\circ$, for some integer $n$.

In all cases, the magnitude of the fitted RM was within a few rad/m$^2$ of the value originally derived for the 2023 ASKAP pulse in \citetalias{2024MNRAS.535..909D}.
The RM appears to decrease slightly over the course of the 2023 ASKAP pulse, and \emph{increase} slightly over the course of the two 2024 MeerKAT components; however, all values are consistent with a constant RM across all three pulses.
The per-pulse weighted averaged values (with weighting proportional to the Stokes I profile) are $962 \pm 1$\,${\rm rad}/{\rm m}^2$ (2023 ASKAP), $965 \pm 2$ (2024 ASKAP) and $962 \pm 2$\,${\rm rad}/{\rm m}^2$ (2024 MeerKAT).
The spectral index of the linear component, $\alpha_L$, is similarly consistent, with averages value of $-2.1 \pm 0.4$ (2023 ASKAP), $-3.0 \pm 0.6$ (2024 ASKAP) and $-1.9 \pm 0.5$ (2024 MeerKAT).

Each pulse was de-Faraday rotated with the weighted average RM given above, to derive the linear polarisation profile and the PA evolution across the pulses.
For the 2023 ASKAP pulse, we recover the `S'-shaped PA curve discussed at length in \citetalias{2024MNRAS.535..909D} in the context of the RVM.
Contrastingly, the PA curve of the 2024 ASKAP pulse is flat (across five 10-second bins), and that of the 2024 MeerKAT pulse is curved, but not in an RVM-like way.

\section{Discussion} \label{sec:discussion}

After accounting for dispersion and scattering effects, the timing analysis of \src{} confirms its status as an LPT, as originally conjectured in \citetalias{2024MNRAS.535..909D}.
It has the fifth largest period (${\sim}1.16\,$hours) of known published LPTs, after ASKAP J1839$-$0756 \citep[$6.45\,$hours;][]{Lee2025}, GLEAM-X J0704$-$37 \citep[$2.92\,$hours;][]{2024arXiv240815757H, 2025arXiv250103315R}, ILT J1101+5521 \citep[$2.09\,$hours;][]{deRuiter2025}, and GCRT J1745$-$3009 \citep[$1.28\,$hours;][]{2005Natur.434...50H}.

With the timing solution in hand, we can explore the efficacy of the original follow-up campaign described in \citetalias{2024MNRAS.535..909D}.
Assuming \src{} remained at a similar brightness (see Fig. \ref{fig:pulse_details}), it should have been detected in two ASKAP observations (2023-04-06 and 2023-10-18) and a few dozen times throughout the MWA's 2022 GPM run (2022-06-02 to 2022-09-08).
There are also three observations in the earliest part of the 2024 GPM run (one on 2024-06-02 and two on 2024-06-10) in which no pulse was observed.
However, pulses were detected just ten days later on 2024-06-20 (the first of the MWA 200\,MHz points in Fig. \ref{fig:pulse_details}), and at each epoch thereafter whenever the MWA was observing when a pulse was due to arrive.

We conclude, therefore, that \src{} is intermittent and that the active periods have durations on the order of months.
Even when the source is ``on'', it can apparently change in brightness by an order of magnitude or more (at a given frequency) in different months.
This is reminiscent of the behaviour of GLEAM-X J162759.5$-$523504.3, which was observed over a three-month period, but has not been seen again \citep{2022Natur.601..526H}.
The reappearance of \src{} after so many months suggests the possibility that other known intermittent LPTs (like GLEAM-X J162759.5$-$523504.3 and GCRT J1745$-$3009) may eventually do likewise.

As noted in \citetalias{2024MNRAS.535..909D}, optical follow-up of \src{} is challenging due to its proximity to the Galactic plane ($b = -0.12^\circ$).
Inferring this nature of the system (white dwarf vs neutron star, isolated vs binary) from the radio pulses alone is challenging, unless either the population of LPTs is better understood as a whole, or if some other property of the radio signal (e.g. intermittency, polarisation) can be linked to the system properties.

\src{} shares several properties similar to ILT J1101+5521 and GLEAM-X J0704$-$37, which are confirmed polars.
\src{}'s period of ${\sim}1.16\,$hours is marginally shorter than the minimum known orbital period of $1.3\,$hours \citep{schwope2025polarcatcatalogpolarslowaccretion}, making it a good candidate for testing the suggestion put forward by \citet{2025arXiv250103315R} that ``short LPTs'' ($P \lesssim 78\,$mins) and ``long LPTs'' ($P \gtrsim 78\,$mins) represent AM CVns and CVs respectively.


\dots

%%%%% Discussion of MeerKAT precursor %%%%%
%The precursor burst seen with MeerKAT is apparently unique, and is not visible at other telescopes and frequencies, even after averaging them together to make a single profile.
%The only burst of comparable S/N, and at a similar frequency, in which the dimmer precursor component might be expected to be visible is the original 2023-01-21 ASKAP burst \citepalias{2024MNRAS.535..909D}, where again it is absent.
%...

%%%%% Discussion of brighter pulses in 2023 %%%%%
%The two pulses observed in January 2023 (the original ASKAP pulse and the MWA (IPS) pulse observed the following day) are significantly wider than all other pulses, suggesting that a morphological change occurred between ...

\begin{enumerate}
        \item Why does the source appear to not be active all the time at higher frequencies?
              \item As per convo with John (see MWA Users Slack, 2025-01-24), point out that (lack of detection of source in) MWA IPS is not very constraining in terms of scattering disk / timescale / screen.
\end{enumerate}

\begin{deluxetable*}{lcl}
  \tablecaption{Known and candidate LPTs, arranged in ascending order of period\label{tbl:lpts}}
  \tablehead{
    \colhead{LPT} & \colhead{Period} & \colhead{References} \\ 
     & \colhead{(min)} & 
   } 
   \startdata
   %AR Scorpii (J162147.28-225310.3) & 1.97 &  \\
   %ASKAP J175154.89-255135.3 & ${\sim}$3.6 &  \\
   %J1912-4410 & 5.3 &  \\
   CHIME J0630+25             & 7.02 & \citet{2024arXiv240707480D} \\
   GLEAM-X J162759.5-523504.3 & 18.2 & \citet{2022Natur.601..526H} \\
   GPM J1839-10               & 22.0 & \citet{2023Natur.619..487H} \\
   ASKAP J1832-0911           & 44.3 & \todo{Wang} \\
   ASKAP J193505.1+214841.0   & 53.8 & \citet{2024NatAs...8.1159C} \\
   ASKAP J175534.9-252749.1   & 69.8 & \citetalias{2024MNRAS.535..909D} \\
   GCRT J1745-3009            & 77.0 & \citet{2005Natur.434...50H} \\
   ILT J1101+5521             & 126  & \citet{2024arXiv240811536D} \\
   GLEAM-X J0704-37           & 175  & \citet{2024arXiv240815757H} \\
   ASKAP J1839-0756           & 387  & \citet{Lee2025} \\
   \enddata
\end{deluxetable*}


\section{Conclusions} \label{sec:conclusions}

We have shown that \src{} is in fact an LPT, as conjectured in \citetalias{2024MNRAS.535..909D}).
Its period of ${\sim}1.16\,$hours places it 
\begin{itemize}
  \item Predict another spurt of inactivity in 2025
\end{itemize}

\begin{acknowledgments}
 N.H.-W. is the recipient of an Australian Research Council Future Fellowship (project number FT190100231).
 
 This scientific work uses data obtained from Inyarrimanha Ilgari Bundara, the CSIRO Murchison Radio-astronomy Observatory. Support for the operation of the MWA is provided by the Australian Government (NCRIS), under a contract to Curtin University administered by Astronomy Australia Limited. ASVO has received funding from the Australian Commonwealth Government through the National eResearch Collaboration Tools and Resources (NeCTAR) Project, the Australian National Data Service (ANDS), and the National Collaborative Research Infrastructure Strategy.
% Skipped because it is acknowledged below.
% We acknowledge the Pawsey Supercomputing Centre which is supported by the Western Australian and Australian Governments.
% ASKAP -- I have fixed this to use the correct name for the observatory
The Australian SKA Pathfinder is part of the Australia Telescope National Facility which is managed by CSIRO. Operation of ASKAP is funded by the Australian Government with support from the National Collaborative Research Infrastructure Strategy. ASKAP and the MWA use the resources of the Pawsey Supercomputing Centre. Establishment of ASKAP, Inyarrimanha Ilgari Bundara, and the Pawsey Supercomputing Centre are initiatives of the Australian Government, with support from the Government of Western Australia and the Science and Industry Endowment Fund. We acknowledge the Wajarri Yamaji People as the Traditional Owners and Native Title Holders of the observatory site.

% MeerKAT
The MeerKAT telescope is operated by the South African Radio Astronomy Observatory, which is a facility of the National Research Foundation, an agency of the Department of Science and Innovation.
  % PTUSE (https://skaafrica.atlassian.net/wiki/spaces/ESDKB/pages/1591672833/User+Supplied+Equipment+USE#PTUSE)
  Observations made use of the Pulsar Timing User Supplied Equipment (PTUSE) servers at MeerKAT which were funded by the MeerTime Collaboration members ASTRON, AUT, CSIRO, ICRAR-Curtin, MPIfR, INAF, NRAO, Swinburne University of Technology, the University of Oxford, UBC and the University of Manchester.  The system design and integration was led by Swinburne University of Technology and Auckland University of Technology in collaboration with SARAO and supported by the ARC Centre of Excellence for Gravitational Wave Discovery (OzGrav) under grant CE170100004.


\end{acknowledgments}

%% To help institutions obtain information on the effectiveness of their 
%% telescopes the AAS Journals has created a group of keywords for telescope 
%% facilities.
%
%% Following the acknowledgments section, use the following syntax and the
%% \facility{} or \facilities{} macros to list the keywords of facilities used 
%% in the research for the paper.  Each keyword is check against the master 
%% list during copy editing.  Individual instruments can be provided in 
%% parentheses, after the keyword, but they are not verified.

%\vspace{5mm}
%\facilities{HST(STIS), Swift(XRT and UVOT), AAVSO, CTIO:1.3m,
%CTIO:1.5m,CXO}

%% Similar to \facility{}, there is the optional \software command to allow 
%% authors a place to specify which programs were used during the creation of 
%% the manuscript. Authors should list each code and include either a
%% citation or url to the code inside ()s when available.

%\software{astropy \citep{2013A&A...558A..33A,2018AJ....156..123A},  
%          Cloudy \citep{2013RMxAA..49..137F}, 
%          Source Extractor \citep{1996A&AS..117..393B}
%          }

%% Appendix material should be preceded with a single \appendix command.
%% There should be a \section command for each appendix. Mark appendix
%% subsections with the same markup you use in the main body of the paper.

%% Each Appendix (indicated with \section) will be lettered A, B, C, etc.
%% The equation counter will reset when it encounters the \appendix
%% command and will number appendix equations (A1), (A2), etc. The
%% Figure and Table counter will not reset.

\appendix

\section{The effect of scattering on timing}
\label{app:scattering_dm}

It can be shown that pulses scatter-broadened by a thin screen will appear to the observer as the original pulse convolved with a one-sided exponential kernel with an associated timescale $\tau$ [CITATION].
For intrinsically Gaussian pulses of scale $\sigma$, the result of this convolution is well described by the \textit{exponentially modified Gaussian} (EMG), as given in Eq. \eqref{eqn:emg}.

In the regime $\tau \ll \sigma$, the effect of scattering is negligible and the scattered pulse resembles the original pulse.
ToAs can be obtained in the usual way, i.e., using a matched filter ``template'' constructed from the average pulse profile.

On the other hand, in the regime $\tau \gg \sigma$, the leading edge of the EMG rises sharply (resembling the error function, $\erf$) and the trailing edge resembles the exponential kernel itself.
For this reason, pulse ToAs are typically mapped to the rising edge of highly scattered pulses, which closely approximates the centre of the original unscattered pulse.

In this appendix, we discuss the effect of scattering on measuring a DM from pulse observations if scattering is not taken into account.
As will be shown, the effect is most dramatic when $\tau \approx \sigma$, and in the case of wide, highly scattered pulses from LPTs like \src{}, can lead to a discrepant DM measurement of several hundreds of pm\,cm$^{-3}$.

We consider four methods for defining ToAs:
\begin{subequations}
(1) the position where the leading edge reaches half-maximum (LEHM),
\begin{equation}
    \emg(\ToA{LEHM}) = \frac{1}{2} A\exp\left(-\frac{1}{2}\left(\frac{\mu - t_m}{\sigma}\right)^2\right),
\end{equation}
where
\begin{equation*}
    t_m = \mu - \sqrt{2}\sigma\erfcx^{-1}\left(\frac{\tau}{\sigma}\sqrt{\frac{2}{\pi}}\right) + \frac{\sigma^2}{\tau^2}
\end{equation*}
is the mode of the EMG,
(2) the inflection point on the leading edge (IPLE),
\begin{equation}
    \left.\dd{\emg(t)}{t}\right|_{t = \ToA{IPLE}} = 0,
\end{equation}
(3) the position of the peak of the pulse (PEAK),
\begin{equation}
    \ToA{PEAK} = t_m,
\end{equation}
and (4) the maximum of the convolution of the pulse with a Gaussian profile with scale $\sigma$ (TMPL)
\begin{equation}
    \left.\deriv{}{t}\left(\emg(t) \ast \exp\left[ -\frac{1}{2} \left(\frac{t - \mu}{\sigma}\right)^2 \right]\right)\right|_{t = \ToA{TMPL}} = 0.
\end{equation}
The last definition is akin to template matching with a high-frequency (negligibly scattered) pulse profile.
\end{subequations}

None of the methods is entirely accurate in the presence of scattering.
We define the errors of the ToAs to be the difference between the measured ToA and the mean of the unscattered pulse,
\begin{equation}
    \Delta \ToA{} \equiv \ToA{} - \mu.
\end{equation}
For LEHM and IPLE, the ToA is measured to be earlier than $\mu$ ($\Delta{\rm ToA} < 0$); for PEAK and TMPL, later ($\Delta{\rm ToA} > 0$).
$\ToA{LEHM}$ and $\ToA{IPLE}$ will perform much better than $\ToA{PEAK}$ and $\ToA{TMPL}$ highly scattered pulses ($\tau \gg \sigma$), whereas the converse is true when scattering is minimal ($\tau \ll \sigma$).
All methods, however, perform relatively poorly in the intermediate regime when $\tau \approx \sigma$.

\begin{figure}[th]
    \centering
    \includegraphics[width=0.98\linewidth]{scattering_DM.png}
    \caption{Analysis of DM measurement errors if scattering is not properly taken into account. Panel (a) shows the errors measured for each defined type of ToA, normalised to the scale of the original pulse, $\sigma$. In panel (b), the predicted error in DM is shown if the ToA error (compared to a high-frequency ToA measurement) is erroneously attributed to dispersion. Panel (c) shows the rate of change of the ToA error with frequency, and panel (d) shows the error in DM measured if this rate of change (the ``slope'' of the pulse across the observed dynamic spectrum) is erroneously attributed to dispersion. In panels (c) and (d), the derived DM errors assume a pulse scale of $\sigma = 25\,$s and $\nu_s = 230\,$MHz, as estimated for \src{}. In all panels, the blue curves indicate \emph{negative} ToA errors (i.e. the measured ToAs are earlier than the pulse's unscattered mean, $\mu$), and the red curves indicate positive quantities.}
    \label{fig:scattering_DM}
\end{figure}

We computed numerically \citep[using SciPy's \texttt{root} function;][]{2020NatMe..17..261V} the errors for the four ToA types defined above, in the regime around $\tau \approx \sigma$.
These are shown in the panel (a) of Fig. \ref{fig:scattering_DM}, normalised to the scale of the pulse, $\sigma$.
The axis along the top shows the normalised timescale (inverted, with largest timescales on the left), while the axis along the bottom shows the equivalent frequencies normalised to $\nu_s$, defined as the frequency at which $\tau = \sigma$.
We have assumed a scattering index of $-4$, such that
\begin{equation}
    \frac{\tau}{\sigma} = \left(\frac{\nu}{\nu_s}\right)^{-4}.
\end{equation}

A set of ToAs may be converted to DM measurements in two ways: (1) by comparing the ToAs to those measured at much higher frequencies (e.g. with TMPL) where the scattering is known to be negligible, and (2) by measuring the slope of the pulse in the dynamic spectrum (for example, by measuring the ToAs in subbands across the observed frequency range).
Panel (c) of Fig. \ref{fig:scattering_DM} shows the expected slope measurements for the ToA errors given in the panel (a).

In the first case, the ToA error of a pulse measured at frequency $\nu$, if attributed erroneously to the effect of dispersion, will yield a discrepant DM of
\begin{equation}
    \Delta{\rm DM}_{\rm hi-freq} \approx \frac{\Delta\ToA{}\,\nu^2}{\mathcal{D}}.
\end{equation}
For the specific case of \src{}, for which we estimate $\sigma = 25\,$s and $\nu_s = 230\,$MHz, consistent with $\tau_{\rm sc,1 GHz} = 70\,$ms, the DM error can be as high as a few hundreds of pc\,cm$^{-3}$.
When using this DM method, it is better to use LEHM or IPLE at frequencies $\nu \lesssim \nu_s$, and PEAK or TMPL for $\nu \gtrsim \nu_s$; however, it should be remembered that LRHM and IPLE will \emph{underestimate} the DM, while PEAK and TMPL will \emph{overestimate} it.
All ToA definitions produce sizeable DM errors in the approximate range $\nu_s \lesssim \nu \lesssim 3\nu_s$.

The situation is not much improved for DMs determined via the slope of the pulse across the observed band, for which the scattering-induced slope is mapped to an (erroneous) equivalent DM measurement of
\begin{equation}
    \Delta{\rm DM}_{\rm in-band} \approx \deriv{{\rm ToA}}{\nu} \frac{\nu^3}{2\mathcal{D}}
\end{equation}
The calculated numbers for \src{} are illustrated in panel (d).
For this case, LEHM and IPLE are always preferred to PEAK and TMPL, but the magnitude of the derived DM error is still in the hundreds of pc\,cm$^{-3}$ at frequencies in the range $\nu_s \lesssim \nu \lesssim 3\nu_s$.

We confirm the order of magnitude of the above results by reporting a measured in-band DM for one of the MWA pulses at 185\,MHz, using the TMPL method to measure ToAs in 1.28\,MHz subbands from ${\sim}170$ to ${\sim}200\,$MHz, of ${\rm DM}_{\rm in-band} = 1221\pm257\,$pc\,cm$^{-3}$.
This exceeds the DM reported in this work by ${\sim}450\,$pc\,cm$^{-3}$, in excess of the prediction shown in panel (d), but consistent within measurement errors.

\bibliography{biblio}{}
\bibliographystyle{aasjournal}

%% This command is needed to show the entire author+affiliation list when
%% the collaboration and author truncation commands are used.  It has to
%% go at the end of the manuscript.
%\allauthors

%% Include this line if you are using the \added, \replaced, \deleted
%% commands to see a summary list of all changes at the end of the article.
%\listofchanges

\end{document}

% End of file `sample631.tex'.

